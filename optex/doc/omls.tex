\fontfam[lm]
\activettchar`
\enlang
\enquotes
\def\labitem[#1]{\label[#1]\wlabel{\the\_itemnum}}
\let\_narrowlastlinecentered=\ignoreit

\tit \OpTeX/ Markup Language Standard

{\it\hfil Petr Olšák, 2021}\bigskip

The \OpTeX/ markup language standard (OMLS) declares a list of control
sequences used in \OpTeX/ documents including their syntax and sematic. The
listed control sequences in OMLS are called {\it known} and other control
sequences are {\it unknown}.

The main reason for OMLS is to give instructions on how to program convertors
from \OpTeX/ documents to another formats (Html, Markdown, \LaTeX/) or how
to interpret the \OpTeX/ document sources in such applications as
`texcount` or color highlighting in editors\fnote
{or advanced editor features: hidding/uncovering sections/chapers,
 auto-completions, etc.}.
These converters and applications are called {\it cnv-programs} in this document.

We suppose that if you need absolute control over the typography of the
document when it is converted to PDF pages, then you use \OpTeX/ itself. If
you need to create other formats of the same document then you use a
cnv-program which accepts this OMLS. The result is a document without
typographical instructions like dimensions of pages, margins, paginations
and headers, selection of a font-family, dimensions of the fonts, etc.\ You can
imagine the result of such a conversion as one Html page where more
typographical features can be controlled in a different way, for example by
an external CSS file. This is a reason why control sequences like `\fontfam` or
`\margins` are ignored by cnv-programs.

Obviously, \TeX/ and \OpTeX/ itself gives possibility to declare various
input formats for various purposes. Sometimes (in very special cases) there
exists a good reason to declare a different and special input format by
\TeX/ macros. But if the source of the document respects the OMLS then it is
reasonably transformable to other formats by cnv-programs. We hope that
OMLS-ready documents cover a very large set of typical documents used these
days.

We suppose that cnv-programs work internally with strings of source lines
without tokenization. This is one of the great difference in processing the
document directly by \OpTeX/ and using a cnv-program. The second difference is
that the expansion process of macros is not implemented in cnv-programs in
its full range. We respect that the result of cnv-programs will be different
than from processing directly by \OpTeX/. But this is not a bug, this is
the feature. We concentrate on the fixed syntax and sematic given by OMLS of the
\OpTeX/ document and we throw behind the head the typographical
exactness of the document which can be done only directly by \TeX/ (and it is
exactly described in \TeX/book, for example).

\sec Syntactical elements

The syntactical elements are described as strings here. No \TeX/'s tokenization is
taken into account. The rule with a smaller number has a precedence.
It means for example, that `\%` is <singlechar-control sequence> by rule~\ref[singlecs]
and it does not start the comment by rule~\ref[commentline].

\begitems \style n
* <eol> is end of line.
* `%%:` at beginning of the line means a <cnv-declarator>, see section~\ref[cnv-decl].
* <cnv-declarator><text><eol> is interpreted specially.
* <space> is a space or a tab-character.
* <spaces> is a non-empty sequence of <space>s.
* a line only with <spaces> is <empty-line>.
* <letter> is a character `a`--`z` or `A`--`Z`.
* <specletter> is a  <letter> or `_`.
* <singlechar-control-sequence> is `\`<non-specletter>.\labitem[singlecs]
* <letters-seq> is a non-empty sequence of <specletter>s.
* <multiletter-control-sequence> is `\`<letters-seq> finalized by a <non-specletter> is
* <comment> is `%`<text><eol> and it should be ignored including <eol>.\labitem[commentline]
* <spaces> at beginning of the line says that the line is {\it indented}.
* <spaces><eol> or <eol> should be interpreted as a single <space>, <eol> is removed.
* If a <multileter-control-sequence> is finalized by <space>, the <space> is ignored.
* <control-sequence> is <multiletter-control-sequence> or <singlechar-control-sequence>
* `{`<balanced text>`}` is text where each next inner `{` must match with an inner `}`.
* `{`<balanced text>`}` is interpreted as a parameter followed by a <control-sequence>.
* The single `{` not used by previous rule opens a group and the single `}` closes the same group.
* There are two modes: h-mode, v-mode\fnote
  {this is great simplification of real \TeX/ modes.}.
  The document processing starts in v-mode.
* The swithing from v-mode to h-mode means: a paragraph begins.
* The swithnig from h-mode to v-mode means: the current paragraph ends.
* The <spaces> between words in h-mode should be replaced by single space.
* In v-mode: a single <non-space> character or control sequences listed
  in table~\ref[vtohmode] swithes to h-mode.
* In h-mode: an empty line or control sequences listed in
  table~\ref[htovmode] switches to v-mode.
* <number> is a non-empty sequence of digits with optional + or $-$ in the front.
* <decimal-number> is <number> with optional dot inside the sequence of digits.
* <tex-unit> is a pair of letters listed in the table~\ref[tex-units]
  preceded by an optional <space>.
* <dimen> is <decimal-number><tex-unit>.
* <unknown-control-sequence> is <control-sequence> not listed in section~\ref[listcs].
* <unknown-control-sequence>`=`<number> should be ignored including its parameter.
* <unknown-control-sequence>`=`<dimen> should be ignored including its parameter.
* <unknown-control-sequence>`={`<blanced-text>`}` should be ignored including its parameter.
* The optional <space> can be present immediately after `=` in previous three rules.
* <unknown-control-sequence>`{`<blanced-text>`}` is processed as <blanced-text> alone.

\enditems

\bigskip
\label[vtohmode]\caption/t
List of control sequences which switch from v-mode to h-mode.
\cskip
\begblock
`\indent`, `\noindent`, `\leavevmode`, `\hskip`, `\hfil`, `\hfill`, `\hss`,
`\`<space>, `\vrule`.
\endblock

\bigskip
\label[htovmode]\caption/t
List of control sequences which switch from h-mode to v-mode.
\cskip
\begblock
`\par`, `\vskip`, `\vfil`, `\hrule`, `\bigskip`, `\medskip`, `\smallskip`,
`\chap`, `\sec`, `\secc`, `\secl`, `\end`, `\bye`, `\begitems`, `\begtt`,
`\begblock`, `\enditems`, `\endblock`.
\endblock

\bigskip
\label[tex-units]\caption/t
List of \TeX/ units.
\cskip
\begblock
`pt`, `pc`, `bp`, `dd`, `cc`, `in`, `cm`, `mm`, `sp`, `em`, `ex`.
\endblock

\sec Declaration and text parts of the document

... todo

\sec[listcs] List of known control sequences

... todo

\sec Math mode processing

... todo

\sec[cnv-decl] The `%%:` declarators

... todo

\sec Notes on various conversions

... todo

\bye
