\fontfam[lm]
\activettchar`
\enlang

\tit \OpTeX/ Markup Language Standard

{\it\hfil Petr Olšák, 2021}\bigskip

The \OpTeX/ markup language standard (OMLS) declares a list of control
sequences used in \OpTeX/ documents including their syntax and sematic. The
listed control sequences in OMLS are called {\it known} and other control
sequences are {\it unknown}.

The main reason for OMLS is to give instructions on how to program convertors
from \OpTeX/ documents to another formats (Html, Markdown, \LaTeX/) or how
to interpret the \OpTeX/ document sources in such applications as
`texcount` or color highlighting in editors. These converters and applications
are called {\it cnv-programs} in this document.

We suppose that if you need absolute control over the typography of the
document when it is converted to PDF pages, then you use \OpTeX/ itself. If
you need to create other formats of the same document then you use a
cnv-program which accepts this OMLS. The result is a document without
typographical instructions like dimensions of pages, margins, paginations
and headers, selection of a font-family, dimensions of the fonts, etc.\ You can
imagine the result of such a conversion as one Html page where more
typographical features can be controlled in a different way, for example by
an external CSS file. For example, control sequences `\fontfam` or
`\margins` are ignored by cnv-programs.

Obviously, \TeX/ and \OpTeX/ itself gives possibility to declare various
input formats for various purposes. Sometimes (in very special cases) there
exists a good reason to declare a different and special input format by
\TeX/ macros. But if the source of the document respects the OMLS then it is
reasonably convertible to other formats by cnv-programs. We hope that
OMLS-ready documents cover a very large set of typical documents used these
days.

We suppose that cnv-programs work internally with strings of source lines
without tokenization. This is one of the great difference in processing the
document directly by \OpTeX/ and using a cnv-program. The second difference is
that the expansion process of macros is not implemented in cnv-programs in
its full range. We respect that the result of cnv-programs will be different
than from processing directly by \OpTeX/. But this is not a bug, this is
the feature. We concentrate on the fixed syntax and sematic given by OMLS of the
\OpTeX/ document and we throw behind the head the typographical
exactness of the document which can be done only directly by \TeX/ (and it is
exactly described in \TeX/book, for example).

\sec Syntactical elements

... todo

\sec Declaration and text parts of the document

... todo

\sec List of known control sequences

... todo

\sec Math mode processing

... todo

\sec Configuration possibilities of cnv-programs

... todo

\bye
