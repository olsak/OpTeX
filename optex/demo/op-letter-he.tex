%% This is part of the OpTeX project, see http://petr.olsak.net/optex
\useOpTeX

\def\נושא{\subjec}
\pagedirection 1 \bodydirection 1 \pardirection 1 \textdirection 1 % all directions r2l
\letter            % \letter OpTeX style activated
\helang \quotes % Language and quote style
\fontfam[culmus]


\hfill\address
 הארי פוטר
 פְּרִיוֶוט 4 
וינגינג תחתית, סארי

\bigskip

\address
  אלבוס דמבלדור
  הוגוורטס
  סקוטלנד

\hfill\today

\נושא מכתב קבלה ל\"הוגוורטס"

 אדון וגברת דַרְסְלִי, דיירי דרך פְּרִיוֶוט מספר ארבע, ידעו לדווח בגאווה שהם נורמליים לגמרי---ותודה ששאלתם. לא יעלה על הדעת כי מכל האנשים בעולם דווקא הם יסתבכו בפרשיות מוזרות או מסתוריות, והרי הם פשוט לא סובלים שטויות מסוג זה.

מר דַרסְלי היה מנכ"ל של חברה בשם גְרַאנִינְגְס לייצור מקדחות. הוא היה איש גדל־ממדים, בשרני, וכמעט נטול צוואר---למרות שדווקא היה לו שפם שמן למדי. גברת דַרסְלי היתה רזה ובלונדינית, ולה היה צוואר ארוך פי־שניים מהאורך המקובל, מה שהיה שימושי מאוד, כי רוב זמנה עבר עליה בהצצה מעל גדרות כדי לרגל אחר השכנים שלה. לַדַרסְלים היה תינוק ששמו דַאדְלִי, ובעיניהם לא היה בעולם ילד מוצלח ממנו.

דבר לא היה חסר לדַרסְלים. אולם היה להם סוד, והם חיו בפחד שמישהו יגלה אותו. הם חשבו שחייהם לא יהיו חיים ברגע שמישהו ישמע על משפחת פּוֹטֶר. גברת פּוֹטר היתה אחותה של גברת דַרסְלי, אבל עברו שנים רבות מאז נפגשו לאחרונה. למעשה, גברת דַרסְלי נהגה להעמיד פנים כאילו בכלל אין לה אחות, כי אחותה ובעלה הכלומניק היו האנשים הכי לא דַרסְליים בעולם. הדַרסְלים הצטמררו מלחשוב מה יגידו השכנים אם יום אחד הפּוֹטרים יבואו לבקר ברחוב שלהם. הדַרסְלים ידעו שגם לַפּוֹטרים נולד ילד קטן, אבל הם מעולם לא ראו אותו. הילד היה סיבה טובה נוספת להתרחק מהפּוֹטרים---הם לא רצו שדאדְלי יתחבר עם ילד כזה.
\bigskip % leave some space for signature
\bigskip

הארי פוטר
\bye
